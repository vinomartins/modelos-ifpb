\documentclass[11pt,addpoints,answers]{exam}

\def\titulo{Nome da avaliação} % Avaliação I, II, reposição...
\def\disciplina{Disciplina} % Nome da disciplina
\def\curso{Curso - Período}
\def\data{} % substituir por \today para data de hoje
\def\professor{Vinicius Martins} % Substitui por seu nome
% "ifpb.png" se a logo estiver no mesmo diretório
% "../ifpb.png" se a logo estiver no diretório pai
\def\caminhologo{ifpb.png} 


\usepackage{multicol}
\usepackage{tikz}
\usepackage{tcolorbox}
\usepackage[utf8]{inputenc}
\usepackage{amsmath, amsthm, amssymb}


\usepackage{pgfplots}
\pgfplotsset{compat=1.8}
\usepgfplotslibrary{statistics}

\usepackage{graphicx}
\newcommand{\uvec}[1]{\boldsymbol{\hat{\textbf{#1}}}}

\newcommand{\ds}{\displaystyle}
\newcommand{\mdc}{\operatorname{mdc}}
\newcommand{\mmc}{\operatorname{mmc}}
\totalformat{Total da questão \thequestion: \totalpoints}

\hyphenpenalty 10000
%\usepackage[paperheight=5.8in,paperwidth=8.27in,bindingoffset=0in,left=0.8in,right=1in,
%top=0.7in,bottom=1in,headsep=.5\baselineskip]{geometry}
\flushbottom
\usepackage[normalem]{ulem}
\renewcommand\ULthickness{2pt}   %%---> For changing thickness of underline
\setlength\ULdepth{1.5ex}%\maxdimen ---> For changing depth of underline
\renewcommand{\baselinestretch}{1}
\pagestyle{empty}

\newcommand{\dx}{\operatorname{d}\!x}

\pagestyle{headandfoot}
\headrule
\newcommand{\continuedmessage}{%
\ifcontinuation{\footnotesize Question \ContinuedQuestion\ continues\ldots}{}%
}
\runningheader{\footnotesize }
{\footnotesize \disciplina}
{\footnotesize Página \thepage\ de \numpages}
\footrule
\footer{\footnotesize Bom Trabalho!}
{}
{\ifincomplete{\footnotesize Question \IncompleteQuestion\ Continua na próxima página \ldots}{\iflastpage{\footnotesize Fim da atividade}{\footnotesize Continua na próxima página \ldots}}}
%
\usepackage{cleveref}
\crefname{figure}{figure}{figures}
\crefname{question}{question}{questions}
%
%
%==============================================================
\begin{document}
%%
%% \thispagestyle{empty}
%
\noindent
\begin{minipage}[l]{.1\textwidth}%
\noindent
\includegraphics[width=\textwidth]{\caminhologo}
\end{minipage}%
%
\hfill
\begin{minipage}[r]{.68\textwidth}%
\begin{center}
 {\large \bfseries \titulo\ -\ \disciplina  \par}
 {\rm \curso}
\end{center}
\end{minipage}%
%
\fbox{\begin{minipage}[l]{.175\textwidth}%
{\bf Prof.º}\\
\noindent
{ \professor }\\
{\footnotesize Data: {$\rule{.6cm}{0.15mm}/\rule{.6cm}{0.15mm}/\rule{.6cm}{0.15mm}$}}
\end{minipage}}%
%
\vspace{0.2cm}
\par
\emph{\bf Nome\ :\ \underline{\hspace{10cm}}\hspace{0.5cm}
          Nota\ :\ \ \qquad   / 100}
\par
\noindent
\uline{ \hfill}
%

\pointsdroppedatright
\boxedpoints
\pointsinmargin
\phantom{.}

\begin{questions}

  \question[20]
  Um grupo de estudantes foi submetido a um teste de matemática e os resultados foram
  condensados na seguinte tabela de frequência:
  \begin{table}[h]
    \centering
    \begin{tabular}{c|c}
      Nota & Frequência \\ \hline
      $0 \vdash 2$ & 14 \\
      $2 \vdash 4$ & 28 \\
      $4 \vdash 6$ & 27 \\
      $6 \vdash 8$ & 11 \\
      $8 \vdash 10$ & 4 \\
    \end{tabular}
  \end{table}
  \begin{itemize}
    \item[a)] Construa um histograma das notas obtidas.
    \item[b)] Se a nota mínima para ter direito a recuperação é $4$, qual
      é a porcentagem dos alunos que \textbf{não} obtiveram nota suficiente 
      ter direito a recuperação?
    \item[c)] Se a nota mínima para a aprovação direta é $7$, é possível calcular
      a porcentagem dos alunos que foram aprovados diretamente? Justifique. Caso seja impossível,
      você consegue estimar esse valor (por exemplo, apresentar um intervalo que contenha
      o valor real).
  \end{itemize}

  \question[15]
  Sejam $A$ e $B$ dois eventos em um dado espaço amostral,
  tais que $P(A) = .2$, $P(B) = p$, $P(A\cup B) = .5$ e $P(A \cap B) = .1$.
  Determine o valor de $p$.

  \question[15]
  Analise a afirmação a seguir:
  \begin{tcolorbox}
  Estatísticas do Departamento de Trânsito do Rio Grande do Sul informam que
  cerca de $44\%$ dos condutores mortos em acidentes de trânsito nesse estado
  em 2022 estavam alcoolizados, de forma que cerca de $56\%$ estavam sóbrios.
  Dessa forma é possível concluir que é mais seguro conduzir alcoolizado já
  que, dado que um condutor morreu em um acidente, é mais provável que ele
  estivesse sóbrio do que alcoolizado.

  \end{tcolorbox}
  Você concorda com a afirmação? Caso não concorde explique por que ela estaria
  errada.
  \question
  Um médico desconfia que um paciente tem um tumor no abdômen, pois isto
  ocorreu em $\alpha = 60\%$  dos casos similares que tratou. Para auxiliar
  no diagnóstico, 
  recomendou ao paciente que fizesse um ultrassom. Se o paciente de fato
  tiver o tumor, o exame de ultrassom o detectará com uma probabilidade
  de $\beta = .90$. Entretanto, se ele não tiver o tumor, o exame pode,  
  erroneamente, 
  indicar que o tem com probabilidade de $\gamma = .15$.  
  \begin{parts}
    \part[15] Quais são as taxas de falso positivo e falso negativo do exame?
    \part[20] Se o exame detectou o tumor, qual é a probabilidade do paciente
    tê-lo de fato?
    \part[15] Sem refazer os cálculos, o que ocorreria com o a probabilidade do
    item $b)$ nos seguintes casos: 1) $\alpha$ diminuisse? 2) $\gamma$
    aumentasa?
  \end{parts}

  \question[30]
  A probabilidade de um jogador de basquete acertar uma cesta em um
  lance livre é deve $p$. Encontre o menor valor de $p$ para que
  a probabilidade dele acertar ao menos uma cesta em dois lances livres
  seja de $80\%$  supondo que as probabilidades no primeiro e segundo
  lançamento são independentes.

  \question[20]
  Numa certa população, a probabilidade de gostar de teatro é de um $1/3$,
  enquanto a de gostar de cinema é de $1/2$. Determina a probabilidade de
  gostar de teatro e não de cinema nos seguintes casos:
  \begin{itemize}
    \item[a)] Gostar de teatro e gostar de cinema são eventos disjuntos.
    \item[b)] Gostar de teatro e gostar de cinema são eventos independentes.
    \item[c)] Todos que gostam de teatro gostam de cinema.
    \item[d)] A probabilidade de gostar de teatro e de cinema é de $1/8$.
    \item[e)] Dentre os que não gostam de cinema, a probabilidade de
      não gostar de teatro é de $3/4$.
  \end{itemize}

\end{questions}



\begin{center}
\hqword{Questão}
\hpword{Pontos}
\hsword{Nota}
\gradetable[h][questions]
\end{center}

\end{document}
